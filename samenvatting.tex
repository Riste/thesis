\chapter*{Samenvatting}
\pagestyle{samenvatting}
\addcontentsline{toc}{chapter}{Samenvatting}

Gezien het feit dat audiovisuele archieven steeds toegankelijker worden voor het grotere internetpubliek, wordt het succesvol ophalen van archiefitems een steeds uitdagender taak. Dit wordt voornamelijk veroorzaakt door het ontbreken van adequate annotaties: korte beschrijvingen van een video fragment. Gebruikers kunnen niet vinden waar ze naar op zoek zijn, omdat annotaties niet aanwezig zijn of teveel gericht zijn op
het gebruik door experts. Video tagging games zijn een poging om dit probleem te verlichten. Door de internetcommunity in te schakelen om de video's te taggen wordt het probleem van de schaarste
aan annotaties aangepakt en worden tags gegenereerd die op de gemeenschap zijn gericht. De focus van deze thesis ligt op de rol van gametags in audiovisuele archieven en hun toegevoegde waarde
voor het ophalen van video's.

In het typische audiovisuele archief ecosysteem zullen de gametags naast professioneel samengestelde annotaties bestaan. Om te begrijpen wat gametags kunnen bieden in tegenstelling
tot de annotaties van de documentalisten, bestuderen we de verschillen op twee dimensies. De eerste dimensie is de terminologie: we voeren een kwantitatief onderzoek uit om de terminologische
kloof tussen taggers en de documentalisten in te schatten. De tweede dimensie is de beschrijvende scope, d.w.z. welke aspecten van de video-inhoud worden beschreven en op welk niveau van abstractie en
granulariteit. Een kwalitatief onderzoek is uitgevoerd waarin de beschrijvende omvang van de gametags geanalyseerd worden. Onze analyse toont aan dat er een terminologische kloof is tussen
de spelers van \textit{Waisda?} en de professionals. Dit maakt de gametags een waardevolle toevoeging omdat ze afkomstig zijn van de gebruikers en voor de gebruikers zijn. Gametags kunnen de toegang tot de video's verbeteren in het geval dat de gebruikers van het zoeksysteem en de spelers van de game uit dezelfde gebruikersgroep komen. Verder vinden we dat gametags voornamelijk concrete objecten in de video beschrijven en zelden
scènes. Dit resultaat biedt belangrijke intuïtie over het nut van de gametags voor het ophalen van video's, wat de focus is van de rest van de scriptie.

In deze thesis behandelen we twee prominente zoekscenario's die in de praktijk ontstaan. Het eerste scenario is het ophalen van videofragmenten met visuele verschijningen van objecten. In feite lijkt dit erg veel op de ``instantie'' zoektaak van TRECVID met het verschil dat de zoekopdracht is geformuleerd in tekst en niet door een visueel voorbeeld zoals in TRECVID. Het tweede scenario dat we beschouwen, is zoeken naar videofragmenten die over een bepaald onderwerp gaan.

In hoofdstuk 3 bestuderen we hoe effectief de gametags zijn bij het behandelen van het eerste scenario. Hiervoor ontwerpen we Cranfield-stijl experimenten waarvoor het verzamelen van
``documenten'' vereist is (in dit geval videofragmenten die zijn getagd door gebruikers via \textit{Waisda?}), een set zoekopdrachten die we hebben afgeleid van zoekopdrachten in de praktijk en
relevantieoordelen die aangeven welke ``documenten'' in de verzameling moeten worden geretourneerd voor elke zoekopdracht. De resultaten van de evaluatie laten zien dat gametags
uitermate geschikt zijn voor het ophalen van fragmenten die een bepaald object van belang visueel weergeven. Ze presteren beter dan de andere typen metadata (inclusief professionele annotaties) in
paarsgewijze vergelijking. We stellen ook vast dat er een verschil is in de focus en de reikwijdte van de gametags in vergelijking met de professionele annotaties; gametags zijn meer verfijnd, professionele annotaties gaan over de hele video en beschrijven alleen de belangrijkste onderwerpen. Deze eigenschap van de gametags maakt ze bijzonder aantrekkelijk
voor zakelijke gebruikstoepassingen zoals het vinden en verkopen van ``stock footage''.

In hoofdstuk 4 evalueren we de prestaties van gametags voor actuele zoekopdrachten. De resultaten tonen aan dat onbewerkte, onverwerkte gametags niet goed geschikt zijn voor het ophalen van videofragmenten op basis van het onderwerp. Hoewel de zoekopdrachten een bevredigend aantal antwoorden opleveren, laat de zoekprecisie te wensen over. Dit
wordt voornamelijk veroorzaakt door de aanwezigheid van gametags die niet-relevante onderwerpen suggeren. Het snelle tempo van \textit{Waisda?} resulteert meestal in spelers die alleen taggen wat ze direct zien. Om het zoeken op onderwerp te verbeteren, karakteriseren we de kwaliteit van gebruikerstags als onderwerpbeschrijving en identificeren we verschillende tag-functies (filters) die dienen als indicatoren of game-tags die nuttig zijn voor het ophalen. Onze resultaten tonen aan dat gametags na filtering de ophaalprestaties van een baseline-systeem kunnen nabootsen, waarbij
handmatig metadata worden gebruikt voor zoeken. Een belangrijk gevolg van deze bevinding is dat tagging-spellen een kosteneffectief alternatief kunnen bieden in situaties waarin handmatige
annotatie door professionals te duur is.

De algemene conclusie van deze thesis is dat de gametags het zoeken verbeteren en gemiddeld beter presteren in vergelijking met de handmatig vervaardigde metadata en gesloten bijschriften.
Deze verklaring komt met drie voorbehouden.

Ten eerste zijn de taggers en de zoekers in onze context afkomstig van dezelfde populatie, wat resulteert in een aanzienlijke terminologische overlapping tussen de zoekwoorden die worden
gebruikt voor het zoeken en de tags die aan de video's worden toegeschreven. Wij geloven dat dit een factor is die sterk bijdraagt aan het succes van de gametags voor zoeken in de context van ons
onderzoek. Hoewel het in eerste instantie een beperking lijkt, is dat in de praktijk niet het geval. De terminologie die wordt gebruikt bij het taggen is algemeen genoeg om tegemoet te komen aan de
behoeften van iedereen behalve de ervaren gebruikers die wellicht meer gespecialiseerde (specifieke) termen gebruiken bij het zoeken.

Ten tweede, in deze scriptie hebben we alleen naar keyword-gebaseerde zoekopdrachten gekeken, dus de genoemde verbetering in het zoeken is beperkt tot deze modaliteit. Op het moment van schrijven krijgen andere zoekmodaliteiten, zoals het zoeken op basis van voorbeeld videofragmenten, meer aandacht, maar zoeken op basis van zoekwoorden is nog steeds de meest prominente, vooral op het web.

Ten derde richtten we ons in dit werk op single domain (genre), non-fictie tv-komedie. Bijgevolg zijn de verkregen resultaten gebonden aan dit domein. Uit de kwalitatieve analyse van de tags bleek
echter dat dit meestal algemene concepten zijn die visuele objecten beschrijven en daarom niet zo nauw verbonden zijn met het genre van de video-inhoud die ze beschrijven. Onze verwachting is dat
de gametags goed zullen presteren voor de meeste niet-fictionele tv-genres die een breder publiek aanspreken. Het valt nog te bezien of dit het geval zal zijn voor de fictionele genres waar meer
symboliek en abstracte uitdrukkingen te verwachten zijn. We veronderstellen dat zonder de taggers behoorlijk te instrueren of een niche subcommunity van deskundigen te benaderen, de uitkomst
minder bevredigend zal zijn.